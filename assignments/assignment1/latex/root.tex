% -------------------------------------
% -- Assignment 1: Cooling of Coffee --
% -- Math 3090 ------------------------

% Set document type
\documentclass[11pt,oneside]{extarticle}


\usepackage{amsmath}
\usepackage{amssymb}
\usepackage{amsthm}
%\usepackage{thmtools}
\usepackage{fontspec}
\usepackage[margin=0.5in]{geometry}
\usepackage{inputenc}
%\usepackage{unicode-math}
\usepackage{setspace}
\usepackage{fancyhdr}
\usepackage{garamondx}
\usepackage{graphicx}
\usepackage{hyperref}
\usepackage{float}
\usepackage{listings}
\usepackage{newtxmath}
\usepackage{gensymb}
\usepackage{solarized-light}
\usepackage{xcolor}
%\usepackage{lmodern}

%\usepackage[default]{sourcecodepro}
\usepackage[T1]{fontenc}
%\usepackage{inconsolata}
% Real number symbol
\newcommand{\Real}{\mathbb{R}}
\newcommand{\dprime}{{\prime\prime}}
\newcommand{\Celsius}{\:\degree\mathrm{C}}
\newcommand{\Dollar}{\$}
\newcommand{\percent}{\%}
%\newcommand{\min}{\mathrm{min}\:}

% Set typewriter font to Source Code Pro
%\renewcommand{\ttfamily}{\tiny\sourcecodepro}

% Change enumerator to use letters i.e., a, b, c, ...
\renewcommand{\theenumi}{\alph{enumi}}

\setlength{\belowcaptionskip}{-12pt}
%\setttfont{

% Set paragraph indent to 0cm
\setlength{\parindent}{0cm}
%\setlength{\parskip}{1cm plus4mm minus3mm}
\setlength{\parskip}{0.25cm}

\begin{document}

\section{Changing room temperature}

\par Modify the coffee model to account for a room temperature that starts at $20\Celsius$,
and increases at a constant rate to a maximum $26\Celsius$ in $2$ hours, and then
stays at the maximum temperature. Assume that the initial temperature of the cofee
is $100\Celsius$ and after $10$ minutes, the temperature of the coffee is $90\Celsius$.
Modify the codes and draw the room temperature and coffee temperature in the same
diagram for $6$ hours.

\subsection{Building model}

\par The independent variable of the model is time $t$ in hours. The observed temperature
of the coffee $u_{obs}(t)$ is a function of time, with units$\Celsius$.

\par Let $u_{sur}$ be the surruounding temperature, then $u_{sur}$ is given by

\begin{equation}
    u_{sur}(t) = 
    \begin{cases} 
        20 + 3t & 0 \leq t \leq 2 \\
        26 & t > 2 \\
    \end{cases}.
\end{equation}

\par We use the model presented in class, but modify it to use $u_{sur}(t)$ to
get 

\begin{equation}
    \frac{du}{dt}
    =
    c(u_{sur}(t)-u)
\end{equation}

Now, we discretize the model to get

\begin{equation}
    \frac{u_{k+1} - u_k}{\Delta t}
    =
    c(u_{sur}(t_k) - u_k
\end{equation}

where $t_k = k\Delta t$ is the discrete time, at the $k^{th}$ step.

\newpage

\section{When to add milk?}

\par Do you add the milk to coffee straight away? People often ask wheter it is better
to add the cold milk to a hot cup of coffee straight away or wait for a while to
let it cool down and then add the cold milk.

\par Estimate the optimal time to add cold milk so that the coffee cools down to
the drinkable temperature fastests. Let us assume that the drinable temperature
is $50\Celsius$. Assume that the initial temperature of the coffee is $100\Celsius$
and the insulation of the cup is $0.01282$. Room temperature is fixed at $22\Celsius$.
Is your result reasonable? Explain it briefly. Based on your result, make suggestion
to the people about when to add coffee.


\par You can use and modify the model and codes we developed in the class for the
cooling of the coffee in a constant room temperature. Assume that the temperature
of the coffee is independent on special variables. Make further assumptions if needed.

\newpage

\section{Compound interest}

You invest $\Dollar 100$ in a savings account paying $6\percent$ per year. Let
$y(t)$ be the amount in your account after $t$ years. If the interest rate is
compounded continuously, then $y(t)$ solves the ODE initial value problem

$$
\frac{dy}{dt} = ry
$$

where $r=6\percent=0.06$ and $y(0)=\Dollar 100$.

\begin{enumerate}

    \item What is the analytic solution $y(t)$ to this initial value problem?

        Start with the ensatz $y(t)=Ae^{rt}$, and plug into equation above.

        $$
        \frac{dy}{dt}(t) =
        rAe^{rt} = ry(t)
        $$

        Take ensatz at $t\rightarrow 0$, to determine constant $A$,
        
        $$
        y(0)=Ae^{r\cdot 0} = A
        \implies
        A = y(0) = \Dollar 100.
        $$

        The analytic solution is therefore,

        \begin{equation}
            y(t)=100e^{rt}.
            \label{eq:compound}
        \end{equation}

    \item What is the balance in your account after 10 years with each of the
        following methods of compounding interest, yearly, monthly, quarterly,
        daily, and continuous compounding?

        \par {\bf Compound interest} can be solved analytically using 
        \ref{eq:compound}, as follows:

        $$
        y(10) = 100 e^{0.06\cdot 10} = \Dollar182.21.
        $$

\end{enumerate}


\end{document}
